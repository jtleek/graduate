\documentclass[letterpaper,10pt]{article}
\usepackage[utf8x]{inputenc}

\usepackage{amsmath,amsfonts,amsthm,amssymb}
\usepackage{mathrsfs} %script font
\usepackage[margin=1in]{geometry}

\parindent=0pt
\parskip=10pt
\newcommand{\io}{\;\text{i.o.}}
\renewcommand{\aa}{\;\text{a.a.}}

\def \B {\mathscr{B}}
\def \F {\mathscr{F}}
\def \Q {\mathbb{Q}}
\def \R {\mathbb{R}}
\begin{document}

\thispagestyle{empty}

\textsf{
\begin{flushleft}
\sc James K. Pringle \\
\normalfont 550.620 \\
Dr. Jim Fill \\
Assignment 3 \\
10 October 2012, Wednesday
\end{flushleft}
} \bigskip

\begin{center}
\bf 550.620 Homework \#3 (to turn in)
\end{center}

\begin{enumerate}
\item[(a)] Let $E_1, E_2, \dots$ be an arbitrary sequence of events satisfying
\[
\mbox{(i)} \lim_n P(E_n)=0\qquad \mbox{and} \qquad \mbox{(ii)} \sum_{n} P(E_n\cap E_{n+1}^c)<\infty.
\]
Prove that $P(E_n\io) = 0$.  
\item[(b)] Show that the result of part~(a) strengthens the first Borel--Cantelli Lemma by showing that it implies the first Borel--Cantelli Lemma.
\item[(c)] Deduce that the result of part~(a) \emph{strictly} strengthens the first Borel--Cantelli Lemma by providing an explicit example of a probability space $(\Omega, \F, P)$ and a sequence of events $E_1, E_2, \dots$ such that 
$\sum_n P(E_n) = \infty$ but the result of part~(a) allows us to conclude that $P(E_n \io) = 0$. 
\end{enumerate}

\textit{Solution:}
\begin{enumerate}
\item[(a)]
First we prove a lemma that
\[
\bigcup_{i=n}^m E_n = \left(\bigcup_{i=n}^{m-1} (E_i \cap E_{i+1}^c)\right) \bigcup E_m
\]
by strong induction, starting the index at arbitrary positive integer $n$.
The base case is for $m = n+1$, and the statement is 
$$
\cup_{i=n}^{n+1}E_n = \cup_{i=n}^n(E_i \cap E_{i+1}^c) \cup E_{n+1}
$$
Calculating on the right-hand side, 
\begin{align*}
\cup_{i=n}^n(E_i \cap E_{i+1}^c) \cup E_{n+1} &= (E_n \cap E_{n+1}^c) \cup E_{n+1} \\
&= (E_n \setminus E_{n+1}) \cup E_{n+1} \\
&= E_n \cup E_{n+1} \\
&= \cup_{i=n}^{n+1}E_n
\end{align*}
as desired. Now we move to the inductive step. Suppose the statement is true for all integers $m$ with $n \leq m \leq k$. We show it is true for $m = k+1$. In particular, we show
\[
\cup_{i=n}^{k+1}E_n = \cup_{i=n}^k (E_i \cap E_{i+1}^c) \cup E_{k+1} 
\text{ .}
\]
Working with the right-hand side, we have
\begin{align*}
\cup_{i=n}^k (E_i \cap E_{i+1}^c) \cup E_{k+1} 
&= \cup_{i=n}^{k-1} (E_i \cap E_{i+1}^c) \cup (E_k \cap E_{k+1}^c) \cup E_{k+1} \\
&= \cup_{i=n}^{k-1} (E_i \cap E_{i+1}^c) \cup E_k \cup E_{k+1} \\
&= (\cup_{i=n}^{k-1} (E_i \cap E_{i+1}^c) \cup E_k) \cup E_{k+1} \\
&= (\cup_{i=n}^k E_i ) \cup E_{k+1} 
\text{ by the inductive hypothesis} \\
&= \cup_{i=n}^{k+1} E_i
\end{align*}
as desired. Thus by strong induction, the statement 
\[
\bigcup_{i=n}^m E_n = \left(\bigcup_{i=n}^{m-1} (E_i \cap E_{i+1}^c)\right) \bigcup E_m
\]
is true for all integers $m > n$, and our lemma is concluded.

Now we return to problem (a). Applying the probability function to both sides of the equation from the lemma and calculating a little, we have
\begin{align*}
P(\cup_{i=n}^m E_n)
&= P((\cup_{i=n}^{m-1} (E_i \cap E_{i+1}^c)) \cup E_m) \\
&\leq P(\cup_{i=n}^{m-1} (E_i \cap E_{i+1}^c)) + P(E_m) 
\text{ by subadditivity} \\
&\leq \left(\sum_{i=n}^{m-1} P(E_i \cap E_{i+1}^c)\right) + P(E_m) 
\text{ by subadditivity} \\
&\leq \left(\sum_{i=n}^{\infty} P(E_i \cap E_{i+1}^c)\right) + P(E_m)
\text{ .}
\end{align*}
Now we take the limit as $m$ approaches infinity.
\begin{align*}
P(\cup_{i=n}^\infty E_n) 
= \lim_{m \rightarrow \infty} P(\cup_{i=n}^m E_n)
&\leq \lim_{m \rightarrow \infty} [ \left(\sum_{i=n}^{\infty} P(E_i \cap E_{i+1}^c)\right) + P(E_m) ] \\
&= \sum_{i=n}^{\infty} P(E_i \cap E_{i+1}^c) + \lim_{m \rightarrow \infty} P(E_m) \\
&= \sum_{i=n}^{\infty} P(E_i \cap E_{i+1}^c) \text{ by condition (i).}
\end{align*}
By condition (ii), the tail sum $\sum_{i=n}^{\infty} P(E_i \cap E_{i+1}^c)$ tends down to $0$ as $n$ approaches infinity. Taking the limit as $n$ approaches infinity where we left off, we have
\begin{align*}
0 
= \lim_{n \rightarrow \infty} \sum_{i=n}^{\infty} P(E_i \cap E_{i+1}^c)
&\geq \lim_{n \rightarrow \infty} P(\cup_{i=n}^\infty E_n) \\
&= \lim_n\! \downarrow P(\sup_{k \geq n} E_k) \\
&= P( \lim_n\! \downarrow \sup_{k \geq n} E_k)
\end{align*}
by monotone sequential continuity from above. 
Since $0 \geq P(\displaystyle \lim_n\! \downarrow \sup_{k \geq n} E_k) = P( \limsup_n E_n) = P(E_n \io) \geq 0$, we have $P(E_n \io) = 0$.
\mbox{}~\hfill $\Box$

\item[(b)]
The result of (a) has already been proven. Let $E_1, E_2, \dots$ be an arbitrary sequence of events satisfying $\sum_n P(E_n) < \infty$. It follows that the summand tends to $0$, or in particular  $\lim_n P(E_n)=0$. Thus condition (i) is satisfied. Furthermore, for all $n$, $E_n \cap E_{n+1}^c \subset E_n$, and $P(E_n \cap E_{n+1}^c) \leq P(E_n)$. Therefore,
\[
\sum_n P(E_n \cap E_{n+1}^c) \leq \sum_n P(E_n) < \infty \text{ .}
\]
Hence condition (ii) is satisfied. By (a), we have that $P(A_n \io) = 0$, which is also the conclusion of the first Borel-Cantelli Lemma. Clearly then, the result of (a) implies the first Borel-Cantelli Lemma.
\mbox{}~\hfill $\Box$

\item[(c)]
Let $\Omega = (0,1]$, $\F = \B$, the Borel set on the unit interval, and $P$ be Lebesgue measure. This defines our probability space, $(\Omega, \F, P)$. Define $A_n$ to be the open interval $(0,1/n)$. Clearly $\lim_n P(A_n) = \lim_n (1/n) = 0$. Hence condition (i) holds. $A_n \cap A_{n+1}^c = (0,1/n] \cap (0,1/(n+1)]^c = (0,1/n] \cap (1/(n+1),1] = (1/(n+1),1/n]$. It follows that
\begin{align*}
\sum_{n = 1}^\infty P(A_n\cap A_{n+1}^c) 
= \sum_{n=1}^\infty P((1/(n+1),1/n]) 
&= P((1/2, 1]) + P((1/3,1/2]) + \cdots \\
&= 1 - 1/2 + 1/2 - 1/3 + \cdots \\
&= 1
\text{ .}
\end{align*}
Since the intervals are all disjoint and probability is Lebesgue measure, it is clear that the sum is equal to $1$. Hence condition (ii) holds. Notice that $\sum_n P(A_n) = \sum_n (1/n) = \infty$. However, $P(A_n \io) = 0$ by (a). Therefore (a) strictly strengthens the first Borel-Cantelli Lemma.
\mbox{}~\hfill $\Box$

\end{enumerate}

\end{document}
